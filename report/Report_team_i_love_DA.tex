\documentclass[10pt]{article}
\usepackage[utf8]{inputenc}
\usepackage{geometry}
\usepackage{graphicx}
\usepackage{booktabs}
\usepackage{hyperref}
\usepackage{bookmark}
\usepackage{amsmath}

\geometry{a4paper, margin=0.95in}

\title{Data Analysis Project Report}
\author{
  Team: I love data analysis\\ 
  Peter Felber \& Andreas Heindl \& Jakob Hütter
}

\date{\today}

\begin{document}

\maketitle

\begin{center}
\end{center}

\section{Contributions}
The following contributions were made by each team member:
\begin{itemize}
    \item Peter Felber:
    \begin{itemize}
        \item Data preprocessing tasks
        \item Initial visualization development
    \end{itemize}
    \item Andreas Heindl:
    \begin{itemize}
        \item Statistical analysis implementation
        \item Regression analysis
    \end{itemize}
    \item Jakob Hütter:
    \begin{itemize}
        \item Advanced visualizations
        \item Report writing and documentation
    \end{itemize}
\end{itemize}

\section{Dataset Description}
\begin{itemize}
    \item Dataset name and source: Solar Power Generation Data by Ani Kannal from Kaggle
    \item Time period and sampling frequency: data has been collected over a period of 34 days with a sampling frequency of 15 minutes
    \item Key variables analyzed: DC\_Power, AC\_Power, Ambient Temperature, Module Temperature, Irradiation
    \item Basic statistical properties:
    \begin{itemize}
        \item Number of observations: 3134
        \item Missing values: 130 (should be 34 days * 24 hours * 4 observations per hour = 3264)
        \item Key statistics of cleaned dataset: 
            \begin{table}[h!]
            \centering
            \begin{tabular}{lccccc}
            \toprule
            Variable & Mean & Median & Min & Max & Std \\
            \midrule
            DC\_Power in MW & 67.540 & 8.632 & 0.000 & 269.097 & 85.798 \\
            AC\_Power in MW & 66.060 & 8.344 & 0.000 & 262.392 & 83.858 \\
            Ambient Temperature in $^\circ$C & 25.5 & 24.7 & 21.1 & 33.8 & 3.3 \\
            Module Temperature in $^\circ$C & 31.1 & 24.8 & 19.2 & 60.3 & 12.1 \\
            Irradiation kW/m$^2$ & 0.2273 & 0.0289 & 0.0000 & 0.999 & 0.2950 \\
            \bottomrule
            \end{tabular}
            \label{tab:basic_stats}
            \end{table}
    \end{itemize}
\end{itemize}

\section{Methods and Analysis}

\subsection{Data Preprocessing}
\begin{itemize}
    \item Cleaning procedures: Fix AC\_Power wrong factor to get correct kW values, synchronize Datetime format 
    \item Outlier handling: Replace outliers with missing values, but remove rows with 6 consecutive outliers, to decrease time frame of interpolation
    \item Missing value treatment: Interpolate them with plausible values
    \item Data transformations: Split original dataframe to seperate inverters to different columns
\end{itemize}

\subsection{Exploratory Data Analysis}
\begin{itemize}
    \item Distribution analysis: For total power we can observe a lean towards lower values, possible due to night time. This represents an inverse gaussian distribution. This of course correlates with the IR-Radiation distribution. The ambient temperature shows multi-modal tendencies, with clear bumps around 23 and 28 degrees. For the module temperatures this is less pronounced, with the bumps at 22 and 45 degrees.
    \item Time series patterns:
    \item Correlation analysis:
    \item Key visualizations:
\end{itemize}

\subsection{Statistical Analysis}
\begin{itemize}
    \item Probability analysis: 
    \begin{itemize}
        \item The probability of Total\_AC exceeding the threshold valu of 120 MW is approximately 0.15.
    \begin{itemize}
        \item This value changes: the higher the threshold, the lower the probability.
    \end{itemize}
    \item The cross tabulation analysis shows the distribution of Total\_AC exceeding the threshold across different levels of Irradiation.
    \begin{itemize}
        \item Depending on the set threshold level, the distribution changes. When set to a higher Total\_AC, the more likely it is with a higher irradiation level to be over the threshold.
    \end{itemize}
    \item The conditional probability analysis reveals the likelihood of Total\_AC exceeding the threshold given different Irradiation levels.
    \begin{itemize}
        \item For this analysis, only the probabilities around the threshold have a probability not to be 1 or 0. Higher levels have a probability of 1, and lower levels have a probability of 0.
    \end{itemize}
\end{itemize}
    \item Law of Large Numbers demonstration: The Law of Large Numbers states that as the number of trials increases, the sample mean will tend to be closer to the population mean. In this case, the Law of Large Numbers is demonstrated by calculating the sample mean of Total\_AC exceeding the threshold value of 90,000 for different sample sizes. As the sample size increases, the sample mean tends to be closer to the population mean, which is the probability of Total\_AC exceeding the threshold value of 90,000.
    \item Central Limit Theorem application: The Central Limit Theorem states that the sampling distribution of the sample mean will be approximately normally distributed, regardless of the population distribution, as the sample size increases. In this case, the Central Limit Theorem is applied by calculating the sample mean of Total\_AC exceeding the threshold value of 90,000 for different sample sizes and plotting the sampling distribution. As the sample size increases, the sampling distribution tends to be closer to a normal distribution.
    \begin{itemize}
        \item Q-Q plot analysis (if applicable):
    \end{itemize}
    \item Regression analysis:
    \begin{itemize}
        \item Model selection:
        \item Model fitting and validation:
        \item Cross-validation (if applicable):
    \end{itemize}
\end{itemize}

\section{Key Findings}

\subsection{Statistical Insights}
\begin{itemize}
    \item Distribution characteristics:
    \item Significant correlations:
    \item Probability analysis results:
\end{itemize}

\subsection{Pattern Analysis}
\begin{itemize}
    \item Temporal patterns:
    \item Variable relationships:
    \item Identified anomalies:
\end{itemize}

\subsection{Advanced Analysis Results}
\begin{itemize}
    \item Interactive visualization insights:
    \item Regression performance:
    \item Additional findings:
\end{itemize}

\section{Summary and Conclusions}
\begin{itemize}
    \item Main insights:
    \item Limitations:
    \item Future analysis suggestions:
\end{itemize}

% Optional section if needed
%\section{References}
%\begin{enumerate}
%    \item Reference 1
%    \item Reference 2
%\end{enumerate}

\end{document}